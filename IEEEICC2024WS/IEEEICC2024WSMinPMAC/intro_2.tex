\section{Introduction}

% Wireless communication is critically important in augmented reality (AR) and virtual reality (VR) devices, primarily due to the resource-intensive nature of data processing in these edge devices. AR and VR technologies demand high computational power to render complex, immersive environments in real-time, which can be exceedingly demanding for the processors within the devices. Wireless communication can offload these computationally intensive tasks to powerful servers. This server-side processing approach allows for handling complex calculations and rendering tasks, which are then efficiently transmitted back to the edge device. This not only enhances the performance and capabilities of AR/VR devices but also significantly reduces the hardware requirements and power consumption at the user's end. In this regard, we will introduce the minPMAC algorithm to allocate the optimal data rate to each user based on the energy constraint and their distance to the access point (AP).

% write on XR systems advancement
% XR (Extended Reality), encompassing AR (Augmented Reality), VR (Virtual Reality), and mixed reality, revolutionizes digital content use, merging virtual and real worlds for enhanced experiences in education, healthcare, and entertainment \cite{xr1, xr2, xr3, xr4}.  XR promises the capability to supplant traditional computing devices, positioning itself as a ubiquitous computing platform. The full potential of XR technologies hinges on the development and deployment of low latency, low power, high bandwidth wireless communication systems. Such systems are crucial for delivering seamless, real-time experiences, minimizing delays that can disrupt immersion and user comfort. High bandwidth ensures rich, detailed virtual environments, while low power consumption is essential for the portability and longevity of wearable XR devices, making these technologies more accessible and effective in everyday applications, which is this work's focus.

% % write on distributed rendering problem
% This research addresses distributed rendering's challenge in Extended Reality (XR) systems \cite{rendering1}, with a focus upon Virtual Reality (VR) environments. The general use case has \( N \) users, each equipped with resource-limited VR headsets, engaging in a collaborative VR experience. These users, distributed within a virtual room, observe distinct environmental segments, resulting in individual partial views. Each user transmits these fragmented visual inputs to a central processing server via uplink connections using advanced Wi-Fi technology. The central server, endowed with substantial computational capabilities, employs advanced image processing algorithms to amalgamate these disparate visual fragments into a cohesive three-dimensional spatial room representation. Upon a 3D map's successful synthesis, the server returns this integrated visual content to the users through a high-speed downlink Wi-Fi channel. This process ensures that VR participants ireceive a seamless, uninterrupted 360-degree panoramic view, thereby enhancing the real-time immersive quality of the virtual experience.

This research explores distributed rendering in XR systems, emphasizing VR environments for collaborative experiences. It examines scenarios where users with VR headsets interact in a virtual space, sending visual data to a central server via Wi-Fi. The server processes and integrates these inputs into a unified 3D environment, then redistributes it to users, ensuring a seamless 360-degree view. This approach aims to enhance real-time immersion by leveraging high bandwidth, low latency, and low power wireless communication, crucial for effective and accessible XR applications. The uplink distributed rendering scenario is a multiple-input-multiple-output (MIMO) multiple-access channel (MAC). The multiple-access channel is a fundamental model in communication, characterizing a situation where multiple transmitters are sending information to a single receiver\cite{book}. The uplink MIMO system with additive white gaussian noise (AWGN) can be expressed as:
\begin{equation}
    \boldsymbol{y} = H\cdot \boldsymbol{x} + \boldsymbol{n}
\end{equation}
Where $\boldsymbol{n}$ is the Gaussian noise, and $\boldsymbol{x}$, $\boldsymbol{y}$ are the transmitted and received signals, respectively. 

% write on 500 Mbps problem
Two primary challenges emerge as significant roadblocks for distributed XR rendering: the uplink bandwidth constraint and the edge devices' considerable power consumption, which have inherent resource constraints. XR heavily relies on Head-Mounted Displays (HMDs), which necessitate stringent adherence to power and weight limitations. The imperative to optimize the Quality-of-Experience (QoE) mandates that HMDs be lightweight and compact. Consequently, this necessitates the offloading of substantial computing and storage tasks to external processing units, such as computers or cloud-based servers.  Complex three-dimensional imagery transmission, a cornerstone of these VR applications, necessitates data rates reaching upwards of 500 Mbps or more per user. However, existing wireless methods' capabilities, even those conforming to the latest Wi-Fi standards (802.11b/g/n/ac/be) \cite{ieee80211ax2021}, are insufficient for such high data rate demands, especially when the number of users' antennas exceeds the number of access-point (AP) antennas, and/or the channel is low rank. This issue complicates transmitting uplink 3D image data. Additionally, the current achievable data rates through Wi-Fi, while substantial, lead to prohibitive power consumption levels. This scenario is particularly challenging for devices like Augmented Reality (AR) glasses, where limited power resources are a critical constraint. These limitations significantly impede XR systems' development and practical realization suggest improved real-time data transmission for distributed VR rendering.

% write on prior work to address problem

% write on proposed solution (cite ellipsoid and GDFE papers)
This work addresses these challenges by optimizing the data rates and the power consumption using a non-linear generalized decision feedback equalizer (GDFE) \cite{book}. These systems achieve the maximum possible data rates for each of the \( N \) users in a VR setting, considering factors such as the channel impulse responses between the users and the AP, the number of antennas, the distance from the AP, and the transmit signal-to-noise ratio at each user's device. In a converse approach, given each user's prerequisite minimum data rates, the objective minimizes the edge devices' power consumption. Tthe proposed GDFE-based solutions achieve much higher data rates with much lower power consumption, compared to current Wi-Fi standards, using the time-sharing (vertex sharing) technique. This dual optimization strategy also overcomes the prevailing bottlenecks in uplink communication capacity, thereby paving the way for high-fidelity XR systems' development and deployment.

% write on metrics we outperform on
A presented series of extensive experiments evaluate the proposed system's performance for Extended Reality (XR) applications. These experiments encompass a broad parameter range, which includes channel impulse responses, the number of users involved, the quantity of antennas per user, requisite minimum data rates, signal-to-noise ratio (SNR), and the spatial distances from the access point (AP).  The proposed GDFE-based approach augments the traditional Orthogonal Frequency Division Multiplexing (OFDM)-based Wi-Fi methods.

These GDFE results demonstrate a significant achievable-rate increase, reaching upwards of 500 Mbps per user antenna. This figure notably surpasses the data rates feasible through traditional Wi-Fi methodologies. Furthermore, results find a substantial energy-saving improvement. Specifically, for a given minimum required data rate, a threshold that traditional Wi-Fi can accomplish, the proposed algorithm achieves this benchmark with an order of magnitude lower user-device energy consumption. This finding is particularly impactful, considering the constraints of power resources in edge devices such as Augmented Reality (AR) glasses, commonly employed in XR systems. 