\section{Introduction}

% Wireless communication is critically important in augmented reality (AR) and virtual reality (VR) devices, primarily due to the resource-intensive nature of data processing in these edge devices. AR and VR technologies demand high computational power to render complex, immersive environments in real-time, which can be exceedingly demanding for the processors within the devices. Wireless communication can offload these computationally intensive tasks to powerful servers. This server-side processing approach allows for handling complex calculations and rendering tasks, which are then efficiently transmitted back to the edge device. This not only enhances the performance and capabilities of AR/VR devices but also significantly reduces the hardware requirements and power consumption at the user's end. In this regard, we will introduce the minPMAC algorithm to allocate the optimal data rate to each user based on the energy constraint and their distance to the access point (AP).

% write on XR systems advancement
XR (Extended Reality), encompassing AR (Augmented Reality), VR (Virtual Reality), and mixed reality, is revolutionizing how we interact with digital content, merging virtual and real worlds for enhanced experiences in education, healthcare, and entertainment \cite{xr1, xr2, xr3, xr4}. In a broader perspective, XR harbors the capability to supplant traditional computing devices, positioning itself as a ubiquitous computing platform. The full potential of XR technologies hinges on the development and deployment of low latency, low power, high bandwidth wireless communication systems. Such systems are crucial for delivering seamless, real-time experiences, minimizing delays that can disrupt immersion and user comfort. High bandwidth ensures rich, detailed virtual environments, while low power consumption is essential for the portability and longevity of wearable XR devices, making these technologies more accessible and effective in everyday applications.



% write on distributed rendering problem
In our research, we address the distributed rendering challenge in Extended Reality (XR) systems \cite{rendering1}, with a focus on Virtual Reality (VR) environments. We examine a scenario involving \( N \) users, each equipped with resource-limited VR headsets, engaging in a collaborative VR gaming experience. These users, distributed within a virtual room, observe distinct segments of the environment, resulting in individual partial views. These fragmented visual inputs are then transmitted to a central processing server via uplink connections using Wi-Fi 7 technology. The central server, endowed with substantial computational capabilities, employs advanced image processing algorithms to amalgamate these disparate visual fragments into a cohesive three-dimensional spatial representation of the room. Upon successful synthesis of a complete 3D map, the server redistributes this integrated visual content back to the users through a high-speed downlink Wi-Fi channel. This process ensures that participants in the VR environment receive a seamless, uninterrupted 360-degree panoramic view, thereby enhancing the immersive quality of the virtual experience in real-time.

The uplink distributed rendering scenario is a multiple-input-multiple-output (MIMO) multiple-access channel (MAC). The multiple-access channel is a fundamental model in communication, characterizing a situation where multiple transmitters are sending information to a single receiver. The uplink MIMO system with additive white gaussian noise (AWGN) can be expressed as:
\begin{equation}
    \boldsymbol{y} = H\cdot \boldsymbol{x} + \boldsymbol{n}
\end{equation}
Where $\boldsymbol{n}$ is the Gaussian noise, and $\boldsymbol{x}$, $\boldsymbol{y}$ are the transmitted and received signals, respectively. 


% write on 500 Mbps problem
Two primary challenges emerge as significant roadblocks for distributed XR rendering: the constraints of uplink communication bandwidth \cite{impediment1} and the considerable power consumption at edge devices, which are inherently resource-constrained \cite{impediment2}. XR heavily relies on Head-Mounted Displays (HMDs), which necessitate stringent adherence to power and weight limitations. The imperative to optimize the Quality-of-Experience (QoE) mandates that HMDs be lightweight and compact. Consequently, this necessitates the offloading of substantial computing and storage tasks to external processing units, such as computers or cloud-based servers. The transmission of complex three-dimensional imagery, a cornerstone of these VR applications, necessitates data rates reaching upwards of 500 Mbps per user antenna. However, the capabilities of existing wireless communication protocols, even those conforming to the latest Wi-Fi standards (802.11b/g/n/ac) \cite{wifi1, wifi2}, are insufficient for such high data rate demands, especially when the number of antennas at the users exceeds the number of antennas at the access point (AP), and/or the channel is low rank. This issue makes the task of transmitting uplink 3D image data impractical. Additionally, the current achievable data rates through Wi-Fi, while substantial, lead to prohibitive power consumption levels. This scenario is particularly challenging for devices like Augmented Reality (AR) glasses, where limited power resources are a critical constraint. These limitations significantly impede the development and practical realization of XR systems dependent on intense and real-time data transmission for distributed rendering in VR environments.

% write on prior work to address problem


% write on proposed solution (cite ellipsoid and GDFE papers)
In this paper, we address these challenges by optimizing the data rates and the power consumption using a non-linear generalized decision feedback equalizer (GDFE) \cite{gdfe, yunGlobecom}. {\color{red} write little more on algorithm details} We achieve the maximum possible data rates for each of the \( N \) users in a VR setting, considering factors such as the channel impulse responses between the users and the AP, the number of antennas, the distance from the AP, and the transmit signal-to-noise ratio at each user's device. In a converse approach, given the prerequisite minimum data rates necessary for each user, our objective is to minimize the power consumption at each of these edge devices. We show that the proposed GDFE-based solutions achieve much higher data rates with much lower power consumption, compared to current Wi-Fi standards, using the time-sharing (vertex sharing) technique. Through this dual optimization strategy, we endeavor to overcome the prevailing bottlenecks in uplink communication capacity, thereby paving the way for the development and deployment of high-fidelity XR systems.

% write on metrics we outperform on
We undertake a series of extensive experiments to rigorously evaluate the performance of our proposed distributed rendering system for Extended Reality (XR) applications. These experiments are meticulously designed to encompass a broad range of parameters, including channel impulse responses, the number of users involved, the quantity of antennas per user, requisite minimum data rates, signal-to-noise ratio (SNR), and the spatial distances from the access point (AP). We compare the results of the proposed GDFE-based algorithm to the traditional Orthogonal Frequency Division Multiplexing (OFDM)-based Wi-Fi standards.

Our results demonstrate a significant enhancement in achievable data rates with the implementation of the GDFE-based algorithm, reaching upwards of 500 Mbps per user antenna. This figure notably surpasses the data rates feasible through traditional Wi-Fi methodologies. Furthermore, our analysis reveals a substantial improvement in power efficiency. Specifically, for a given minimum required data rate — a threshold that traditional Wi-Fi can accomplish — our proposed algorithm demonstrates the capability to achieve this benchmark with an order of magnitude lower power consumption on the user devices. This finding is particularly impactful, considering the constraints of power resources in edge devices such as Augmented Reality (AR) glasses, commonly employed in XR systems. These experimental outcomes underscore the superiority of our proposed algorithm, not only in terms of data rate enhancement but also in significantly reducing power consumption. 