\section{Related Work}
% XR papers
Recent studies \cite{xr1, xr2, xr3, xr4} underscore the critical need for low-power uplink wireless communication systems capable of supporting advanced extended reality (XR) applications' high data rates. Traditional approaches, such as orthogonal multiple access (OMA)-based techniques, have been extensively explored in this context. These techniques, including time division multiple access (TDMA) and orthogonal frequency division multiple access (OFDMA) \cite{louie1992multiple}, involve users transmitting either at distinct time intervals or using frequency-domain resource blocks. Although OMA methods' implementations are straightforward and efficient, they do not meet the high data rate demands, particularly uplink 500+ Mpbs requirements that surpass current Wi-Fi standards \cite{ieee80211bRevision, ieee80211ac2013, ieee80211ax2021}.

% noma papers
This limitation has sparked significant research interest in non-orthogonal multiple-access (NOMA) methods as a wireless-transmission alternative. The literature \cite{noma1, noma2, noma3, noma4, noma5} extensively discusses these approaches and emphasizes power-domain NOMA with the AP's superposition coding (SC) use downlink and successive interference cancellation (SIC) based decoding uplink. This method demonstrates a considerable data-rate improvement for a given signal-to-noise ratio (SNR). However, even with these advancements, the data achieved rates represented as 3 bits/s/Hz and translating to 240 Mbps for an 80 MHz channel still do not meet high-quality XR applications' necessary thresholds  \cite{noma1, noma2, noma3, noma4}. Additionally, \cite{noma5} simultaneously transmits and receives reflecting intelligent surfaces (STAR-RISs) in conjunction with NOMA communications. This approach has shown potential in achieving spectral efficiency rates up to 11.2 bits/s/Hz. Despite these promising results, the practical implementation of RISs introduces additional complexities that the community continues to address. 

% old GDFE paper


% specify baseline OFDM paper