\section{Related Work}

% XR papers
Recent studies \cite{xr1, xr2, xr3, xr4} have underscored the critical need for low-power uplink wireless communication systems capable of high data rates to support advanced extended reality (XR) applications. Traditional approaches, such as orthogonal multiple access (OMA)-based techniques, have been extensively explored in this context. These techniques, including time division multiple access (TDMA) \cite{tdma} and orthogonal frequency division multiple access (OFDMA) \cite{ofdm}, involve users transmitting either at distinct time intervals or using orthogonal basis functions across different blocks in the frequency domain. Although OMA methods are straightforward and efficient for implementation, they fall short in meeting the high data rate demands, particularly the requirement of over 500 Mbps per user antenna in the uplink direction, which surpasses current Wi-Fi standards \cite{linear1, linear2, linear3}.

% noma papers
This limitation has sparked significant research interest in non-orthogonal multiple-access (NOMA) methods as an alternative for wireless transmission. The literature \cite{noma1, noma2, noma3, noma4, noma5} has extensively discussed these approaches, with particular emphasis on power domain NOMA, which involves superposition coding (SC) at the user end for uplink transmission and successive interference cancellation (SIC) based decoding at the access point (AP). This method demonstrates a considerable improvement in data rates for a given signal-to-noise ratio (SNR). However, even with these advancements, the data rates achieve—represented as 3 bits/s/Hz and translating to 300 Mbps for a 100 MHz channel—still do not meet the necessary thresholds for high-quality XR applications \cite{noma1, noma2, noma3, noma4}. Additionally, \cite{noma5} has explored the use of simultaneously transmitting and receiving reflecting intelligent surfaces (STAR-RISs) in conjunction with NOMA communications. This approach has shown potential in achieving spectral efficiency rates up to 11.2 bits/s/Hz. Despite these promising results, the practical implementation of RISs introduces additional complexities, making it less feasible for everyday real-time communication systems. 

% old GDFE paper


% specify baseline OFDM paper